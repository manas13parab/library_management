\documentclass[a4paper,12pt]{article}

\usepackage[T1]{fontenc}
\usepackage[utf8]{inputenc}
\usepackage{xcolor}
\usepackage{listings}
\usepackage{geometry}
\geometry{margin=1in}

% Python syntax highlighting style
\lstdefinestyle{python}{
    language=Python,
    backgroundcolor=\color{gray!5},
    keywordstyle=\color{blue}\bfseries,
    stringstyle=\color{red!60!black},
    commentstyle=\color{green!50!black}\itshape,
    basicstyle=\ttfamily\small,
    showstringspaces=false,
    frame=single,
    breaklines=true,
    tabsize=4,
    numbers=left,
    numberstyle=\tiny\color{gray},
    captionpos=b
}

\title{\textbf{Explanation of Library Management System Code}}
\author{Generated by ChatGPT (GPT-5)}
\date{\today}

\begin{document}

\maketitle
\hrule
\vspace{10pt}

\section*{Overview}
This project implements a simple \textbf{Library Management System} in Python.  
It is composed of two main scripts:
\begin{enumerate}
    \item \texttt{utils.py} – defines a helper function to generate unique IDs for books.
    \item \texttt{main.py} – provides an interactive console-based menu that lets users manage the library’s books.
\end{enumerate}

The system can:
\begin{itemize}
    \item Add new books to a collection.
    \item View all available books.
    \item Search for specific books.
    \item Delete unwanted books.
    \item Save data for future sessions.
\end{itemize}

\section*{1. utils.py – Utility Module}

\begin{lstlisting}[style=python, caption={utils.py}]
import random 
import string

def generate_id():
    return ''.join(random.choices(string.ascii_uppercase + string.digits, k=5))
\end{lstlisting}

\subsection*{Explanation}
The \texttt{generate\_id()} function:
\begin{itemize}
    \item Uses Python’s \texttt{random} and \texttt{string} modules.
    \item Combines uppercase letters (\texttt{A–Z}) and digits (\texttt{0–9}).
    \item Randomly picks 5 characters to create an alphanumeric ID.
\end{itemize}

\textbf{Purpose:}  
Each book in the system can have a unique 5-character identifier such as “A7F2K”.  
This helps in easily referencing and managing books without confusion.

\section*{2. Main Program – Library Management System}

\begin{lstlisting}[style=python, caption={main.py (core program)}]
from book_operations import add_book, view_books
from manage_books import search_book, delete_book
from storage import save_data, load_data

def main():
    books = load_data()

    while True:
        print("\n=== Library Management System ===")
        print("1. Add Book")
        print("2. View Books")
        print("3. Search Book")
        print("4. Delete Book")
        print("5. Save & Exit")

        choice = input("Enter your choice: ")

        if choice == '1':
            add_book(books)
        elif choice == '2':
            view_books(books)
        elif choice == '3':
            search_book(books)
        elif choice == '4':
            delete_book(books)
        elif choice == '5':
            save_data(books)
            print("✅ Data saved! Exiting...")
            break
        else:
            print("❌ Invalid choice, try again!")

if __name__ == "__main__":
    main()
\end{lstlisting}

\subsection*{Explanation}
The main program controls the overall workflow:
\begin{itemize}
    \item \textbf{load\_data()}: Loads previously saved book data from a file.
    \item Displays a menu with 5 options for user interaction.
    \item Depending on user input:
    \begin{itemize}
        \item Option 1 → Calls \texttt{add\_book()} to insert new book entries.
        \item Option 2 → Calls \texttt{view\_books()} to display all stored books.
        \item Option 3 → Calls \texttt{search\_book()} to find specific titles or IDs.
        \item Option 4 → Calls \texttt{delete\_book()} to remove unwanted books.
        \item Option 5 → Saves data using \texttt{save\_data()} and exits safely.
    \end{itemize}
\end{itemize}

\subsection*{Program Flow Summary}
\begin{enumerate}
    \item The script starts by loading book data.
    \item A continuous loop presents menu options.
    \item The user makes a choice.
    \item The corresponding function executes.
    \item When “Save \& Exit” is chosen, all data is written to storage and the program ends.
\end{enumerate}

\section*{3. Interaction Between Files}
\begin{itemize}
    \item \texttt{utils.py} provides the ID generation utility that other modules (like \texttt{book\_operations.py}) can use when adding new books.
    \item The main file (\texttt{main.py}) integrates all operations: book handling, storage, and user interaction.
\end{itemize}

\section*{Conclusion}
Together, these scripts form the backbone of a modular Library Management System.  
The structure allows easy maintenance, as new features (like editing books or sorting by author) can be added without modifying the core logic.

\end{document}

